\documentclass[titlepage,a4paper]{article}

\usepackage{a4wide}
\usepackage[colorlinks=true,linkcolor=black,urlcolor=blue,bookmarksopen=true]{hyperref}
\usepackage{bookmark}
\usepackage{fancyhdr}
\usepackage[spanish]{babel}
\usepackage[utf8]{inputenc}
\usepackage[T1]{fontenc}
\usepackage{graphicx}
\usepackage{float}

\pagestyle{fancy} % Encabezado y pie de página
\fancyhf{}
\fancyhead[L]{G\_02\_IS\_II }
\fancyhead[R]{Ingenería del Software II - UPM}
\renewcommand{\headrulewidth}{0.4pt}
\fancyfoot[C]{\thepage}
\renewcommand{\footrulewidth}{0.4pt}

\begin{document}
\begin{titlepage} % Carátula
	\hfill\includegraphics[width=9cm]{logoupm.png}
    \centering
    \vfill
    \Huge \textbf{Ciclo 1 — Gestor de Torneos de Tenis}
    \vskip2cm
    \Large Ingenería del Software II\\
    Séptimo cuatrimestre 2024 
    \vfill
    \begin{tabular}{ | c | c | c | } % Datos del versionado
      \hline
      Identificación: & Modificación & Petición Cambio \\ \hline 
      EPD\_V1\_G02: & Versión Inicial & PC 0 \\ \hline
      EPD\_VX\_G02 & Por Rellenar & PC X \\ \hline
  	\end{tabular}
    \vfill
    \vfill
\end{titlepage}

\tableofcontents % Índice general
\newpage

\section{Introducción}\label{sec:intro} %revisar, texto de prueba
El presente documento reune el estandar para definir el proceso de desarrollo del trabajo práctico de la asignatura Ingenería del Software II, que consiste en la elaboración de un sistema de gestión de torneos de tenis. 

\section{Ciclo de Vida}\label{sec:tipo}

El ciclo de vida seguido en el desarrollo de este proyecto es un cascada, por ciclo. El desarrollo de este proyecto consta de 2 ciclos, que seguirán pauta que se especificará en la siguiente sección.

\section{Fases a realizar}\label{sec:modelo}

Para cada uno de estos ciclos, se seguirán las siguientes fases, con sus correspondientes actividades a desarrollar y productos que generan como resultado.

\subsection{Definición de Requisitos}

\subsubsection{Actividades}

Las actividades a desarrollar en este punto consisten en la formulación de los requisitos de una forma atómica, para facilitar su implementación en el punto 3.3. 

\subsubsection{Productos}

Al finalizar estás actividades, se genera un archivo llamado Especificación de Requisitos Software en formato pdf. Adicionalmente, se generara un archivo adicional llamado Inspección de la Especificación de Requisitos Software, tambien en formato pdf.

\subsection{Estimación y Planificación}

\subsubsection{Actividades}
Las actividades a desarrollar en este punto permiten hacer una gestión eficiente de tiempo, para el desarrollo del resto del proyecto. 
\subsubsection{Productos}
Al finalizar estas actividades, se generaran los archivos llamados Estrategia y Planificacion, ambos en formato pdf, asi como el archivo PlanCalidad, en este mismo formato.

\subsection{Diseño de Alto Nivel} %en caso de hacerlo

\subsubsection{Actividades}

\subsubsection{Productos}

\subsection{Codificación y Pruebas}

\subsubsection{Actividades}
Las actividades desarrollar en esta fase consisten en la implementación definidas en el apartado 3.1, incluyendo las mejoras del mismo en base a errores apreciables durante el desarrollo.
\subsubsection{Productos}
%Definir al terminar ciclo.
Al finalizar esta fase, se generan los archivos de código ...

\subsection{Inspección y Seguimiento}

\subsubsection{Actividades}
Las actividades a desarrollar en esta fase consisten en la revisión exhaustiva de todos los productos que se han ido generando en las fases anteriores, en caso de detectar cualquier fallo, se enmendaría de cara al siguiente ciclo.
\subsubsection{Productos}

Al finalizar estas actividades, se generaran los archivos llamados Seguimiento e InformacionSeguimientoDefectos, ambos en formato pdf. Adicionalmente, se generaran los archivos correspondientes a las peticiones de cambio (PCXX) y a los informes de estado semanal (IFSXX), en formato pdf.


\end{document}
